%%%%%%%%%%%%%%%%%%%%%%%%%%%%%%%%%%%%%%%%%
% The original template (the Legrand Orange Book Template) can be found here --> http://www.latextemplates.com/template/the-legrand-orange-book
%
% Original author of the Legrand Orange Book Template:
% Mathias Legrand (legrand.mathias@gmail.com) with modifications by:
% Vel (vel@latextemplates.com)
%
	% Original License:
% CC BY-NC-SA 3.0 (http://creativecommons.org/licenses/by-nc-sa/3.0/)
%%%%%%%%%%%%%%%%%%%%%%%%%%%%%%%%%%%%%%%%%
 
%----------------------------------------------------------------------------------------
%	PACKAGES AND OTHER DOCUMENT CONFIGURATIONS
%----------------------------------------------------------------------------------------

\documentclass[11pt]{book} % Default font size and left-justified equations

\usepackage{algorithm}
\usepackage{algpseudocode}
\usepackage{graphicx}
\usepackage{caption}
\usepackage{subcaption}
\usepackage{animate}
\usepackage{float}

\usepackage[top=3cm,bottom=3cm,left=3.2cm,right=3.2cm,headsep=10pt,letterpaper]{geometry} % Page margins

\usepackage{xcolor} % Required for specifying colors by name
\definecolor{ocre}{RGB}{52,177,201} % Define the orange color used for highlighting throughout the book

% Font Settings
\usepackage{avant} % Use the Avantgarde font for headings
\usepackage{mathptmx} % Use the Adobe Times Roman as the default text font together with math symbols from the Sym­bol, Chancery and Com­puter Modern fonts
\usepackage{mathtools}
\usepackage{microtype} % Slightly tweak font spacing for aesthetics
\usepackage[T1]{fontenc} % Use 8-bit encoding that has 256 glyphs
\usepackage{diagbox}
\usepackage{cancel}
\newtheorem{lemma}{Lemma}
% Bibliography
\usepackage[style=alphabetic,sorting=nyt,sortcites=true,autopunct=true,babel=hyphen,hyperref=true,abbreviate=false,backref=true,backend=biber]{biblatex}
\addbibresource{bibliography.bib} % BibTeX bibliography file
\defbibheading{bibempty}{}

\setlength\parindent{0pt}

\usepackage[format=plain,
            labelfont={bf,it},
            textfont=it]{caption}
            
\DeclareMathAlphabet{\mathcal}{OMS}{cmsy}{m}{n}

\setcounter{secnumdepth}{3}
\setcounter{tocdepth}{3}
\input{headers/structure}
\newcommand{\R}{\ensuremath{\mathbb{R}}}
\newcommand{\Prob}{\ensuremath{\mathbb{P}}}
\newcommand{\E}{\ensuremath{\mathbb{E}}}
\newcommand{\N}{\ensuremath{\mathbb{N}}}

\newcommand{\Xc}{\ensuremath{\mathcal{X}}}
\newcommand{\Yc}{\ensuremath{\mathcal{Y}}}
\newcommand{\Zc}{\ensuremath{\mathcal{Z}}}
\newcommand{\Hc}{\ensuremath{\mathcal{H}}}
\newcommand{\Dc}{\ensuremath{\mathcal{D}}}
\newcommand{\Nc}{\ensuremath{\mathcal{N}}}
\newcommand{\Ac}{\ensuremath{\mathcal{A}}}
\newcommand{\Oc}{\ensuremath{\mathcal{O}}}
\newcommand{\Fc}{\ensuremath{\mathcal{F}}}
\newcommand{\Rc}{\ensuremath{\mathcal{R}}}
\newcommand{\Lc}{\ensuremath{\mathcal{L}}}

\newcommand{\x}{\ensuremath{\mathbf{x}}}
\newcommand{\y}{\ensuremath{\mathbf{y}}}
\newcommand{\w}{\ensuremath{\mathbf{w}}}
\newcommand{\X}{\ensuremath{\mathbf{X}}}
\newcommand{\vv}[1]{\ensuremath{\mathbf{#1}}}



\newcommand{\inprod}[2]{\left\langle #1,#2\right\rangle}
\newcommand{\norm}[1]{\left|\left| #1\right|\right|}
\newcommand{\horzbar}{\hbox{---}}
\newcommand{\eps}{\varepsilon}
\newcommand{\indc}[1]{\ensuremath{\mathbbm{1}\left[#1\right]}}


\newcommand{\iid}{\ensuremath{\overset{iid}{\sim}}}
\newcommand{\gerr}[1]{L_{\D}(#1)}
\newcommand{\terr}[1]{L_{S}(#1)}
\newcommand{\mH}{m_{\H}(\eps,\delta)}
\newcommand{\ridge}{\ensuremath{\widehat{\w}^{ridge}_{\lambda}}}
\newcommand{\ols}{\ensuremath{\widehat{\w}^{ols}}}
\newcommand{\lasso}{\ensuremath{\widehat{\w}^{lasso}_{\lambda}}}
\newcommand{\bestsubset}{\ensuremath{\widehat{\w}^{subset}_{\lambda}}}

\newcommand{\trainset}{\ensuremath{S=\left\{\left(\x_i,y_i\right)\right\}^m_{i=1}}}
\newcommand{\Hreg}{\ensuremath{\Hc_{reg}}}


\newcommand{\todo}[1]{{\color{red}#1}}
\newcommand{\algorithmautorefname}{Algorithm}



% References to GitHub code
\newcommand\Git[1]{\href{https://github.com/GreenGilad/IML.HUJI}{#1}}

\newcommand\GitChapterOneExamples{\href{https://github.com/GreenGilad/IML.HUJI/blob/coding/code/code\%20examples/Chapter\%201\%20-\%20Mathematical\%20Basis.ipynb}{Chapter 1 Examples - Source Code}}

\newcommand\GitChapterTwoExamples{\href{https://github.com/GreenGilad/IML.HUJI/blob/coding/code/code\%20examples/Chapter\%202\%20-\%20Linear\%20Regression.ipynb}{Chapter 2 Examples - Source Code}}

\newcommand\GitChapterThreeExamples{\href{https://github.com/GreenGilad/IML.HUJI/blob/coding/code/code\%20examples/Chapter\%202\%20-\%20Linear\%20Regression.ipynb}{Chapter 3 Examples - Source Code}}

\newcommand\GitChapterSevelExamplesPCA{\href{https://github.com/GreenGilad/IML.HUJI/blob/coding/code/code\%20examples/Chapter\%207\%20-\%20Unsupervised\%20Learning\%20-\%20PCA.ipynb}{Chapter 7 Examples - PCA - Source Code}}

\newcommand\GitChapterSevelExamplesKMeans{\href{https://github.com/GreenGilad/IML.HUJI/blob/coding/code/code\%20examples/Chapter\%207\%20-\%20Unsupervised\%20Learning\%20-\%20PCA.ipynb}{Chapter 7 Examples - KMeans - Source Code}}

\begin{document}
\title{Introduction to Machine Learning (67577) - Course Book - Edition 1}

%----------------------------------------------------------------------------------------
%	TITLE PAGE
%----------------------------------------------------------------------------------------

\begingroup
\thispagestyle{empty}
\AddToShipoutPicture*{\put(0,0){\includegraphics[scale=1.25]{cover}}} % Image background
\centering
\vspace*{5cm}
\par\normalfont\fontsize{35}{35}\sffamily\selectfont
\textbf{Introduction to Machine Learning}\\
{\LARGE 67577\\Course Book}\par % Book title
\vspace*{1cm}
%{\Huge Andrea Hidalgo}\par % Author name
\endgroup

%----------------------------------------------------------------------------------------
%	COPYRIGHT PAGE
%----------------------------------------------------------------------------------------

\newpage
~\vfill
\thispagestyle{empty}
%\noindent Copyright \copyright\ 2014 Andrea Hidalgo\\ % Copyright notice
\textsc{\Git{Introduction to Machine Learning (67577) - Course Book}\\\\ The Rachel and Selim Benin School of Computer Science and Engineering, The Hebrew University, Jerusalem}\\\\\\
Written by Gilad Green, Ury \todo{add surname} and Prof. Matan Gavish\\\\
\textit{First release, March 2021} % Printing/edition date

%----------------------------------------------------------------------------------------
%	TABLE OF CONTENTS
%----------------------------------------------------------------------------------------

\chapterimage{toc_header.png} % Table of contents heading image
\pagestyle{empty} % No headers
\tableofcontents % Print the table of contents itself
\cleardoublepage % Forces the first chapter to start on an odd page so it's on the right
\pagestyle{fancy} % Print headers again


\chapter{Ensemble Methods}
    \section{Bias-Variance Trade-off}
        \subsection{Generalization Error Decomposition}
        \subsection{Lab: Bias-Variance Via Decision Trees}
        \subsection{Lab: Bias-Variance Via Polynomial Fitting}
        
    \section{Ensemble/Committee Methods}
        \subsection{Weak-Learnability}
        \subsection{Uncorrelated Predictors}
        \subsection{Correlated Predictors}
        \subsection{Committee Methods In Machine Learning}
    
    \section{Boosting Weak-Learners}
        \subsection{AdaBoost Algorithm}
        \subsection{Gradient Boosting Algorithm}
        \subsection{Lab: Boosting - Image Classification}
        
    \section{Bagging}
        \subsection{Bootstrapping}
            \subsubsection{} % wider use
        \subsection{Bagging Reduces Variance}
        \subsection{Random Forests Bagging and De-correlating Decision Trees}
\chapter{Ensemble Methods}
    \section{Bias-Variance Trade-off}
        \subsection{Generalization Error Decomposition}
        \subsection{Lab: Bias-Variance Via Decision Trees}
        \subsection{Lab: Bias-Variance Via Polynomial Fitting}
        
    \section{Ensemble/Committee Methods}
        \subsection{Weak-Learnability}
        \subsection{Uncorrelated Predictors}
        \subsection{Correlated Predictors}
        \subsection{Committee Methods In Machine Learning}
    
    \section{Boosting Weak-Learners}
        \subsection{AdaBoost Algorithm}
        \subsection{Gradient Boosting Algorithm}
        \subsection{Lab: Boosting - Image Classification}
        
    \section{Bagging}
        \subsection{Bootstrapping}
            \subsubsection{} % wider use
        \subsection{Bagging Reduces Variance}
        \subsection{Random Forests Bagging and De-correlating Decision Trees}
\chapter{Ensemble Methods}
    \section{Bias-Variance Trade-off}
        \subsection{Generalization Error Decomposition}
        \subsection{Lab: Bias-Variance Via Decision Trees}
        \subsection{Lab: Bias-Variance Via Polynomial Fitting}
        
    \section{Ensemble/Committee Methods}
        \subsection{Weak-Learnability}
        \subsection{Uncorrelated Predictors}
        \subsection{Correlated Predictors}
        \subsection{Committee Methods In Machine Learning}
    
    \section{Boosting Weak-Learners}
        \subsection{AdaBoost Algorithm}
        \subsection{Gradient Boosting Algorithm}
        \subsection{Lab: Boosting - Image Classification}
        
    \section{Bagging}
        \subsection{Bootstrapping}
            \subsubsection{} % wider use
        \subsection{Bagging Reduces Variance}
        \subsection{Random Forests Bagging and De-correlating Decision Trees}
\chapter{Ensemble Methods}
    \section{Bias-Variance Trade-off}
        \subsection{Generalization Error Decomposition}
        \subsection{Lab: Bias-Variance Via Decision Trees}
        \subsection{Lab: Bias-Variance Via Polynomial Fitting}
        
    \section{Ensemble/Committee Methods}
        \subsection{Weak-Learnability}
        \subsection{Uncorrelated Predictors}
        \subsection{Correlated Predictors}
        \subsection{Committee Methods In Machine Learning}
    
    \section{Boosting Weak-Learners}
        \subsection{AdaBoost Algorithm}
        \subsection{Gradient Boosting Algorithm}
        \subsection{Lab: Boosting - Image Classification}
        
    \section{Bagging}
        \subsection{Bootstrapping}
            \subsubsection{} % wider use
        \subsection{Bagging Reduces Variance}
        \subsection{Random Forests Bagging and De-correlating Decision Trees}
\chapter{Ensemble Methods}
    \section{Bias-Variance Trade-off}
        \subsection{Generalization Error Decomposition}
        \subsection{Lab: Bias-Variance Via Decision Trees}
        \subsection{Lab: Bias-Variance Via Polynomial Fitting}
        
    \section{Ensemble/Committee Methods}
        \subsection{Weak-Learnability}
        \subsection{Uncorrelated Predictors}
        \subsection{Correlated Predictors}
        \subsection{Committee Methods In Machine Learning}
    
    \section{Boosting Weak-Learners}
        \subsection{AdaBoost Algorithm}
        \subsection{Gradient Boosting Algorithm}
        \subsection{Lab: Boosting - Image Classification}
        
    \section{Bagging}
        \subsection{Bootstrapping}
            \subsubsection{} % wider use
        \subsection{Bagging Reduces Variance}
        \subsection{Random Forests Bagging and De-correlating Decision Trees}
\chapter{Ensemble Methods}
    \section{Bias-Variance Trade-off}
        \subsection{Generalization Error Decomposition}
        \subsection{Lab: Bias-Variance Via Decision Trees}
        \subsection{Lab: Bias-Variance Via Polynomial Fitting}
        
    \section{Ensemble/Committee Methods}
        \subsection{Weak-Learnability}
        \subsection{Uncorrelated Predictors}
        \subsection{Correlated Predictors}
        \subsection{Committee Methods In Machine Learning}
    
    \section{Boosting Weak-Learners}
        \subsection{AdaBoost Algorithm}
        \subsection{Gradient Boosting Algorithm}
        \subsection{Lab: Boosting - Image Classification}
        
    \section{Bagging}
        \subsection{Bootstrapping}
            \subsubsection{} % wider use
        \subsection{Bagging Reduces Variance}
        \subsection{Random Forests Bagging and De-correlating Decision Trees}    
\chapter{Ensemble Methods}
    \section{Bias-Variance Trade-off}
        \subsection{Generalization Error Decomposition}
        \subsection{Lab: Bias-Variance Via Decision Trees}
        \subsection{Lab: Bias-Variance Via Polynomial Fitting}
        
    \section{Ensemble/Committee Methods}
        \subsection{Weak-Learnability}
        \subsection{Uncorrelated Predictors}
        \subsection{Correlated Predictors}
        \subsection{Committee Methods In Machine Learning}
    
    \section{Boosting Weak-Learners}
        \subsection{AdaBoost Algorithm}
        \subsection{Gradient Boosting Algorithm}
        \subsection{Lab: Boosting - Image Classification}
        
    \section{Bagging}
        \subsection{Bootstrapping}
            \subsubsection{} % wider use
        \subsection{Bagging Reduces Variance}
        \subsection{Random Forests Bagging and De-correlating Decision Trees}
\chapter{Ensemble Methods}
    \section{Bias-Variance Trade-off}
        \subsection{Generalization Error Decomposition}
        \subsection{Lab: Bias-Variance Via Decision Trees}
        \subsection{Lab: Bias-Variance Via Polynomial Fitting}
        
    \section{Ensemble/Committee Methods}
        \subsection{Weak-Learnability}
        \subsection{Uncorrelated Predictors}
        \subsection{Correlated Predictors}
        \subsection{Committee Methods In Machine Learning}
    
    \section{Boosting Weak-Learners}
        \subsection{AdaBoost Algorithm}
        \subsection{Gradient Boosting Algorithm}
        \subsection{Lab: Boosting - Image Classification}
        
    \section{Bagging}
        \subsection{Bootstrapping}
            \subsubsection{} % wider use
        \subsection{Bagging Reduces Variance}
        \subsection{Random Forests Bagging and De-correlating Decision Trees}
\chapter{Ensemble Methods}
    \section{Bias-Variance Trade-off}
        \subsection{Generalization Error Decomposition}
        \subsection{Lab: Bias-Variance Via Decision Trees}
        \subsection{Lab: Bias-Variance Via Polynomial Fitting}
        
    \section{Ensemble/Committee Methods}
        \subsection{Weak-Learnability}
        \subsection{Uncorrelated Predictors}
        \subsection{Correlated Predictors}
        \subsection{Committee Methods In Machine Learning}
    
    \section{Boosting Weak-Learners}
        \subsection{AdaBoost Algorithm}
        \subsection{Gradient Boosting Algorithm}
        \subsection{Lab: Boosting - Image Classification}
        
    \section{Bagging}
        \subsection{Bootstrapping}
            \subsubsection{} % wider use
        \subsection{Bagging Reduces Variance}
        \subsection{Random Forests Bagging and De-correlating Decision Trees}

\chapter{Online- and Reinforcement Learning}
\label{chap:online}
\chapter{Deep Learning}

\newpage
\end{document}